\documentclass{article}
\usepackage{amsmath}
\usepackage{fancyhdr}
\usepackage{amsthm}
\usepackage{caption}
\usepackage{amsfonts}
\usepackage{graphicx}
\usepackage{cite}
\usepackage{float}
\theoremstyle{definition}
\newtheorem{theorem}{Theorem}


\linespread{1.3}
\newcommand{\adjustHeight}{\setlength{\abovecaptionskip}{-15pt}} 
\setlength{\belowcaptionskip}{-4pt} 
  
\makeatletter
\renewcommand\@biblabel[1]{}
\setlength{\parindent}{0pt}


\begin{document}


\title{Response to Referee}
\date{}
\maketitle
I thank the Referee for his or her constructive comments.  I have made significant changes in response to points 1, 3, and 4 from the ``concerns'' section of the Referee Report.  These changes will be enumerated in detail below.

\begin{enumerate}
\item ``One of my concerns is with the motivation. The opening paragraph in the Introduction section makes a convincing argument that the occurrence of operational risk events may be correlated in time due to declining internal controls caused by a significant loss event. However, the second paragraph also talks about frequency being correlated with severity, but the intuition (or empirical evidence) is not presented to motivate such conjecture.''
\\
\\
My original example in the introduction captures both the auto-correlation and the correlation between severity and frequency.  However, this was not clear from the text.  I have revised my introduction to include the following verbiage: ``Following a particularly severe loss event, the manpower required to deal with the event may cause a failure in internal control leading to an in-creased likelihood of an additional event.  The occurrence of an event may be the result of declining internal controls, which may indicate an increased likelihood of  additional  events.   In  such  a  scenario  the  severity  of  an  event  impacts  the frequency of future events, driving correlation between frequency and severity distributions.''
\item ``The paper is quite technical, and for an average reader it is easy to get lost in the technicalities. I have not checked the math, but it looks solid.''
\\
\\
See my response to item \ref{item5}.

\item 
``For the severity distribution, the paper uses alpha-stable distributions that are known to exhibit very heavy tails (in addition to their lack of closed-form expression of the density function). I wonder whether the authors can apply their model with more “traditional” severity distributions commonly used in operational risk literature, such as Log-normal, Weibull, and others. Right now, the application of the proposed approach is too narrowly focused, and the analysis would be much richer if a wider variety of severity distributions were used and the results were compared.''
\\
\\
I agree with the Referee.  I have added analysis around the Gamma and Inverse Gaussian distributions as well.  Note that Log-normal and Weibull distributions do not have closed form characteristic functions and so cannot be efficiently used in this modeling framework.  The analysis of these additional distributions has resulted in an expanded ``Severity Distribution'' section and an expanded ``Numerical Implementation'' section including impacts on value at risk for each severity distribution.

\item
``The proposed model leads to increased tail risk that arises from the dependencies between the frequency and severity distributions. Because of this, as shown in Table 1, VaR estimates increase dramatically with the correlation parameter. With this in mind, what incentive would banks have to use the proposed approach if it leads to VaR estimates that are much higher than in the case with zero correlation? Back to my previous point, perhaps VaR would be less sensitive to correlation if other (less heavy-tailed) severity distributions were used.''
\\
\\
I have added verbiage in my conclusions around why firms would want to use this model: ``The
flexibility  of  the  model  allows  firms  to  more  precisely  estimate  the  capital  to hold  for  operational  risk  and  design  superior  controls  to  mitigate  operational risk.   Since  the  LDA  model  is  a  nested  model,  firms  can  also  test  whether the  LDA  modeling  approach  is  robust  to  correlations  between  frequency  and severity and auto-correlations between frequency.''  Additionally, I have tested the impact of the correlation on the other severity distributions.

\item \label{item5}
``The paper as it stands is theoretical. The analysis would benefit significantly if the proposed methods could be tested with real operational loss data (or at least some simulated data) and VaR estimates under the traditional LDA model (zero correlation) and the proposed model were compared and the implications were discussed.''
\\
\\
My paper is intended to be theoretical.  It provides flexible and efficient extensions to the LDA model.  The model will perform at least as well (in sample) as the standard LDA model regardless of the data since the LDA is a nested model.  The output of the model is a distribution.  If I simulated loss data from the model in this paper I would be essentially guaranteed a perfect fit.  



\end{enumerate}



\end{document}
